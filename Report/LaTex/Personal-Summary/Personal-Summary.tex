\documentclass[a4paper,12pt]{ctexart}

% Packages
\usepackage{geometry}
\geometry{left=2.5cm,right=2.5cm,top=2.5cm,bottom=2.5cm}
\usepackage{fancyhdr}
\usepackage{setspace}
\usepackage{xcolor}
\usepackage{hyperref}

% Header and Footer
\pagestyle{fancy}
\fancyhf{}
\setlength{\headheight}{15pt}
\fancyhead[L]{课程设计个人总结}
\fancyhead[R]{Edu-Hub在线教育平台}
\fancyfoot[C]{\thepage}

% Title Setup
\title{\textbf{Edu-Hub在线教育平台} \\ \large 课程设计个人总结与心得体会}
\author{
    班级:软件233 \\
    学号:32306300071 \\
    姓名:杨竣淇
}
\date{\today}

\begin{document}

\maketitle

\section{项目概况}
本次课程设计中,我独立完成了“Edu-Hub在线教育平台”的全栈开发工作。该项目旨在构建一个轻量级、高互动性的在线教学系统,解决传统教学中资源管理分散、师生互动不足的问题。通过本次实践,我成功实现了基于 RBAC 的用户权限体系、递归章节树管理、视频流媒体播放以及基于部门的课程分发复用等核心功能。

\section{技术收获}

\subsection{前端工程化能力的提升}
在前端开发中,我深入实践了 \textbf{Vue 3 + Vite} 的现代开发模式。通过使用 \textbf{Composition API},我学会了如何将复杂的业务逻辑(如课程进度追踪)抽离为可复用的 Composable 函数,极大地提高了代码的可维护性。同时,\textbf{Pinia} 的使用让我对状态管理有了更清晰的认识,摒弃了 Vuex 中繁琐的 Mutation 概念,使数据流向更加直观。此外,\textbf{Tailwind CSS} 的原子化设计理念让我摆脱了传统 CSS 命名的困扰,能够快速构建出响应式且风格统一的界面。

\subsection{国际化与主题定制}
在提升用户体验方面,我实现了前端的国际化与主题切换功能。对于国际化(i18n),我认识到这不仅仅是简单的文本替换,更需要维护一套独立的 JS 语言包变量,将所有界面文案进行抽离和管理。而在主题切换方面,为了支持明暗模式的平滑过渡,我构建了一套基于 CSS 变量(Custom Properties)的样式系统。通过改变根节点的 CSS 变量值,系统能够实时响应主题变化,这比传统的加载多套 CSS 文件方式更加高效且易于维护。

\subsection{后端架构与数据库设计}
在后端层面,我掌握了基于 \textbf{Node.js (Express)} 构建 RESTful API 的规范。特别是在数据库设计方面,通过 \textbf{Sequelize ORM} 进行模型定义和关联查询,我深刻理解了关系型数据库中“一对多”、“多对多”关系的实际应用。例如,在设计“课程-章节”的树形结构时,我对比了“邻接表”与“嵌套集”模型的优劣,最终选择了适合本项目的邻接表方案,并通过递归算法实现了高效的树形数据组装。

\section{问题解决与挑战}

\subsection{递归组件与数据结构}
在开发“章节编辑器”时,我遇到了多级目录渲染和拖拽排序的难题。初期尝试直接在模板中循环渲染,导致代码极其臃肿且难以维护。后来通过查阅资料,我采用了 Vue 的\textbf{递归组件}技术,配合后端返回的树形 JSON 数据,优雅地实现了无限层级的目录展示。这一过程让我明白了数据结构在前端视图渲染中的决定性作用。

\subsection{权限控制的细粒度实现}
如何确保“学生只能看已选课程”、“教师只能管理自己创建的课程”是系统安全的重点。我设计了基于 \textbf{JWT} 的无状态认证机制,并编写了多个 Express 中间件(如 \texttt{requireTeacher}, \texttt{isEnrolled})进行请求拦截。在接口开发初期,由于系统采用了严格的 JWT 鉴权,后续接口的调试必须依赖有效的 Token。为此,我利用 \textbf{Apifox} 的环境变量功能,配置了自动提取 Token 的后置脚本,实现了登录后自动将 Token 注入全局变量。这一配置打通了接口调用的鉴权链路,极大地提高了开发效率,也帮助我及时发现并修复了潜在的越权漏洞。

\subsection{需求变更与开发范围的权衡}
在单人全栈开发的过程中,我深刻体会到了需求分析与实际开发之间的差距。起初我认为设计好的数据库表结构是完美的,但在后续开发中,随着业务逻辑的深入,我不得不频繁地变更表结构。这让我明白了数据库设计是一个持续迭代的过程,而非一劳永逸。

此外,受限于单人开发的时间和精力,我不得不对原定的功能版图进行大幅裁剪。原计划包含“课程系统”、“论坛系统”和“账户管理系统”三大模块。最终,为了保证核心业务的质量,我完全舍弃了独立的“账户管理系统”,将其简化为基础的个人信息修改;对于“论坛系统”,我也做出了妥协,取消了全站自由发帖的“自由论坛”功能,转而专注于与教学紧密相关的“课程章节评论”功能。这种“做减法”的决策过程虽然痛苦,但却保证了项目核心功能的完整性和稳定性。

\section{心得体会}

\subsection{全栈思维的重要性}
本次项目让我从单一的前端或后端视角跳脱出来,建立了全局的\textbf{全栈思维}。在定义 API 接口时,我会下意识地考虑前端调用的便利性;在设计前端页面时,我也会考虑到后端数据的查询效率。这种换位思考的能力,极大地降低了前后端联调时的沟通成本。

\subsection{规范化与复用的价值}
在项目后期,我尝试基于 Edu-Hub 快速衍生出“企业培训系统”。得益于前期良好的代码规范和模块化设计(如封装通用的 \texttt{CourseCard} 组件、后端的 \texttt{Scope} 查询作用域),我仅用了极短的时间就完成了新系统的原型。这让我深刻体会到,软件工程中的“高内聚、低耦合”不仅仅是一句口号,更是提升开发效率和软件生命周期的金科玉律。

\subsection{结语}
总的来说,这次课程设计不仅是一次技术的练兵,更是一次工程实践的洗礼。它让我明白了,优秀的代码不仅要能运行,更要易于阅读、易于维护、易于扩展。在未来的学习和工作中,我将继续保持对技术的热情,不断探索,精益求精。

\end{document}
